%!TEX root = ./main.tex
``On the Internet, nobody knows you're a dog,'' the famous New Yorker comic from 1993 goes~\cite{NewYorkerDog}.  But this lighthearted early meme belied a significant problem that the digital world brought:  how do you know that you're interacting with a real human online? Or, even better, a human of the appropriate age or with the correct credentials?  

The first major advancement in so-called \emph{proofs of personhood}, even if it was not called as such, was the seminal CAPTCHA paper~\cite{EC:ABHL03} (Eurocrypt '03).  CAPTCHAs relied on the Turing Test~\cite{Turing1950} being hard for at least some problems; the main idea was to provide a (potential) person with a short task that should be very easy for a human but hard for a machine.  For a while, CAPTCHAs were very effective at stopping spam and bots from performing various nefarious activities on the web.

However, with the advancements in machine learning today, CAPTCHAs are essentially completely broken.  There is a large line of recent work on breaking all kinds of CAPTCHA systems using LLMs and other ML techniques~\cite{CCS:DOLZZL25,USENIX:TLLLSH25,plesner2024breaking}.  There have even been documented attacks on websites using AI tools to break CAPTCHAs, such as Akirabot~\cite{Akirabot}, which is an OpenAI-based tool used to target 420,000 sites with spam by bypassing CAPTCHAs.  We believe it should be clear that we can no longer rely on humans being ``smarter'' than bots or agents for proofs of personhood. 

So how can we prove personhood in a world where bots are as smart as humans?  Most solutions today are very ad-hoc and actively detrimental to user privacy.  While we could fill a paper with such solutions, we will only focus on a few here.  As an example, mobile driver's licenses (mDLs) are becoming more popular in the United States.  However, to prove (active) personhood, the standard allows the driver's license verifier to query the Department of Motor Vehicles (DMV) server for a revocation check!  This means the DMV can potentially see and record every verification done, which is a huge privacy issue~\cite{mDLIW,mDLBio}.  There is a large list of organizations and experts that have spoken out against this practice~\cite{NoPhoneHome}, but so far there has been little concrete action by mDL issuers.

These privacy risks are not imaginary. Discord recently was compromised and had over 70,000 government IDs of users stolen~\cite{DiscordStolen}.  Very recently, Discord sparked backlash by announcing that all users would have to verify their age using a government ID to see any kind of ``adult'' content~\cite{DiscordBacklash}--their way of proof of personhood.  Why should users trust Discord with their personal information at this point? 

The one institution that has most directly tried to build a proof of personhood has been World, with Worldcoin~\cite{WorldCoinWhitePaper}.  However, their global biometric-based system has received a considerable amount of backlash.  For instance, EPIC~\cite{EPICStatementWC} stated, ``Worldcoin is a potential privacy nightmare... Mass collections of biometrics like Worldcoin threaten people’s privacy on a grand scale, both if the company misuses the information it collects, and if that data is stolen... We urge regulatory agencies around the world to closely scrutinize Worldcoin.''  Regulators everywhere have taken note:  governments in Europe like Spain and Bavaria have required World to delete user biometric data~\cite{SpainBavariaDeleteWC}, with the Bavarian government~\cite{BavariaGDPRWC} stating, ``as a result of our investigation of the processing of the Worldcoin Foundation, we find that the Worldcoin Foundation has violated the General Data Protection Regulation (GDPR) as described in detail below.''  China has issued warnings against Worldcoin~\cite{ChinaWarningWC}, and Hong Kong regulators~\cite{HongKongWC} ordered WorldCoin to cease all operations in the country, dubbing its data collection as ``unnecessary and excessive.''  Finally, even places like Kenya, which in theory should benefit most from Worldcoin, have had it shut down, with a reporter~\cite{KenyaWC} noting ``From the Constitution to statutory law, every imaginable violation occurred,'' about the operation of Worldcoin in Kenya. This motivates the following important question:
\begin{center}
\emph{How does one prove personhood online?}
\end{center}

We note that none of these systems rigorously formalize what a proof of personhood actually is, or give cryptographic guarantees around it.  We will obviously do this, but before we do, we want to point out that this sort of notion appears in a wide variety of other applications, often masquerading as a different kind of problem.

One of the most notorious and potentially dangerous cyberattacks in recent years was the so-called XZUtils attack~\cite{xzUtilsFAQ,xzUtilsStory}.  A maintainer, calling themselves Jia Tan, targeted an undermaintained project critical to the Linux kernel, and started contributing.  After this individual (or group of individuals) became an accepted maintainer, they uploaded malware to the codebase.  This malware was barely caught at the last second, mitigating what could have been an extremely severe compromise in the Linux kernel.  This sort of attack is becoming widespread even outside of open source software. In particular, North Korean tech operatives frequently attempt to use ML tools to gain remote work jobs for tech companies in the US and Europe~\cite{CyberscoopNK}. According to Mandiant~\cite{PoliticoNK}, nearly every Fortune 500 CIO interviewed admitted to hiring a North Korean engineer.

There are other issues with open-source code projects too:  unscrupulous developers are contributing so-called ``AI slop'' to projects in a highly automated fashion, burdening project maintainers with reviewing large amounts of poorly written code.  This has become enough of a problem that, very recently, Mitchell Hashimoto, the CEO of Hashicorp, personally built a reputation system for maintainers called Vouch~\cite{HashiVouch} to aim at solving this problem, which has gotten quite a bit of attention. Unfortunately, this system lacks many of the guarantees we need, including privacy, but is a start in the right direction.
So:
\begin{center}
\emph{How does one prove (reputable) personhood online?}
\end{center}

The Internet enabled unprecedented communication, collaboration, and commerce. However, this progress always implicitly relied on our ability to distinguish humans from machines and, equally importantly, to verify the identity of the human on the other side of a digital interaction. %The ability to authenticate digital identities and relationships is foundational to leveraging the Internet for meaningful economic, social, and civic activity. Without it, digital interactions cannot reliably support trust-dependent applications.
%Yet current approaches to building this trust are fragile and structurally inadequate. 
Today’s mechanisms for establishing this trust online, range from passwords to centralized identity providers, and both have proven insufficient. Password-based systems, despite their simplicity, create systemic vulnerabilities: services storing passwords become high-value targets for attackers; credential reuse across platforms is widespread; and large-scale data breaches remain persistent and inevitable. Centralized identity providers introduce additional structural risks, concentrating power and creating single points of failure.
Moreover, centralized systems prevent trust from being portable across contexts. A driver’s reputation on a ridesharing platform cannot be transferred to a vacation rental service, even when the underlying trust relationship is genuine. These lock-in effects represent a substantial loss of value for users and fragment trust across digital ecosystems.
This motivated the following question:

\begin{center}
\emph{How does one prove (authenticated) personhood online?}
\end{center}
In summary, our thesis is a \uline{scalable, Sybil-resistant, decentralized proof of reputable and authenticated personhood} can serve as the basis of a trust layer for the Internet. 

\subsection{Our Contribution: Cryptography for Proof of Personhood}
In this work, we combine \emph{personhood credentials (PHCs)} and \emph{verifiable relationship credentials (VRCs)} and efficient zero-knowledge proofs on top of them to form a coherent system for a proof of personhood.  

PHCs are issued by trusted authorities and provide evidence that an individual is a real, unique human. These authorities can include educational institutions issuing student IDs, governments issuing passports or driver’s licenses, employers certifying employment, or ride-sharing platforms issuing driver credentials. Any organization can serve as a \emph{credential-issuing authority}, provided it ensures that each person receives at most one PHC and follows revocation protocols when necessary. PHCs may also encode additional structured information, such as age or role, to support specific applications.  

VRCs are issued peer-to-peer between individuals and capture real-world interactions or trust relationships. For example, a passenger and driver on a ride-sharing platform may issue VRCs to one another after a completed trip. VRCs can also carry endorsements or qualitative feedback, such as a colleague vouching for reliability or a landlord attesting to tenant trustworthiness.  

By combining multiple PHCs from different authorities and VRCs from peers, we create a robust representation of identity, personhood, and other relevant attributes. Crucially, individuals need not reveal all their credentials to every verifier. Using the zero-knowledge proofs~\cite{GolMicRac89}, individuals can prove properties such as being over a certain age or having sufficient endorsements without exposing underlying personal data. This approach allows verification to be privacy-preserving, cryptographically sound, and usable at scale.

An ecosystem of users, companies, and even AI agents with these credentials would provide the backbone of a decentralized trust layer that could enable the online world to continue to grow and flourish in the age of AI.


\paragraph{Our Results.} We make the following contributions. First, we formalize a proof-of-personhood system in the real/ideal world paradigm by defining an ideal functionality $\idealPoPwithUnlink$. The functionality captures the issuance of personhood credentials (PHCs), the creation of verifiable relationship credentials (VRCs), and the ability to prove predicates over collections of such credentials. It provides:  (i) \emph{soundness}, ensuring that any valid show proof must be supported by underlying credentials and attestations;  (ii) \emph{privacy}, ensuring that only information explicitly specified by the predicate is revealed; and  (iii) \emph{unlinkability}, ensuring that credential issuance and attestation interactions cannot be linked across issuers or contexts beyond what is explicitly revealed.  

We further extend the model to capture imperfect PHC issuance by incorporating global validation oracles. These oracles model external validation procedures that may err with bounded probability, allowing us to reason about settings in which issuers can be deceived by adversarial users. In this model, when a show proof aggregates credentials and attestations from multiple independent issuers, the confidence in the resulting claim increases as additional honest sources are incorporated.

Finally, we give a protocol realizing the functionality. To do so, we introduce a new primitive, \emph{vouchable credentials}, which augment standard credentials with the ability to produce succinct proofs tied to revealed attributes. We obtain the following.

\begin{theorem}[informal]
	Assuming the security of pseudorandom functions, non-interactive zero-knowledge proofs, and vouchable credentials, there exists a protocol that securely realizes $\idealPoPwithUnlink$ in the presence of global validation oracles.
\end{theorem}

We also consider a weaker functionality $\idealPop$ in which unlinkability guarantees are removed, capturing deployments where persistent identities are acceptable and efficiency is prioritized.

\begin{theorem}[informal]
	Assuming the security of non-interactive zero-knowledge proofs and vouchable credentials, there exists a protocol that securely realizes $\idealPop$ in the presence of global validation oracles.
\end{theorem}

While vouchable credentials can be constructed generically from any credential scheme together with non-interactive zero-knowledge proofs, we additionally present a concrete instantiation designed for efficiency that makes only black-box use of underlying cryptographic primitives.

\begin{theorem}[informal]
	There exists a vouchable credential scheme in the generic group model that makes only black-box use of cryptographic primitives.
\end{theorem}

We also provide experimental evaluation results, implementing the construction using both our black-box vouchable credential scheme and general-purpose zero-knowledge succinct non-interactive arguments of knowledge (zkSNARKs).

\paragraph{Roadmap.} In the remainder of the paper, we present a technical overview in \cref{sec:overview}, then define the personhood oracle and formal security model in \cref{sec:oracle,sec:definition}. We next introduce our abstraction of vouchable credentials in \cref{sec:vouchable-creds}. We provide both a generic instantiation of vouchable credentials in \cref{sec:generic-vouchable-const}, and a black-box construction in \cref{sec:blackbox-vouchable-creds}. Using vouchable credentials as an underlying tool, we present our proof-of-personhood constructions (base and unlinkable variants) in \cref{sec:construction-base,sec:construction-unlinkable} and conclude with experimental results in \cref{sec:evaluation}.


% \arka{Maybe also add a paragraph for future work. Maybe talk about revocation (referencing the remark in the main body). Multiple hops in an efficient manner etc.}
